\documentclass[12pt,a4paper]{article}

% ---------------------------------------------------------
% Packages
% ---------------------------------------------------------
\usepackage[utf8]{inputenc}
\usepackage{geometry}
\usepackage{graphicx}
\usepackage{fancyhdr}
\usepackage{amsmath, amssymb}
\usepackage{parskip}
\usepackage{array}
\usepackage[table]{xcolor}
\usepackage{hyperref}
\usepackage{colortbl}

% ---------------------------------------------------------
% Page layout
% ---------------------------------------------------------
\geometry{
	top=2.5cm,
	bottom=2.5cm,
	left=2.3cm,
	right=2.3cm
}

\setlength{\parskip}{0.7em}
\setlength{\parindent}{0pt}
\setlength{\headheight}{62pt}  
\addtolength{\topmargin}{-30pt}

% Increase headheight to fit logos + bar
\setlength{\headheight}{60pt}

% ---------------------------------------------------------
% Colors
% ---------------------------------------------------------
\definecolor{headerblue}{RGB}{0,90,170}
\definecolor{headergray}{gray}{0.9}

% ---------------------------------------------------------
% fancyhdr rules
% ---------------------------------------------------------
% Remove default header rule (this was the line under the blue bar)
\renewcommand{\headrulewidth}{0pt}
% We'll draw the footer rule manually
\renewcommand{\footrulewidth}{0pt}

% ---------------------------------------------------------
% User-editable contest info
% ---------------------------------------------------------
\newcommand{\ContestSeriesName}{Programming Contest League}
\newcommand{\ContestTitle}{Sample University Programming League 2025}

% ---------------------------------------------------------
% Full-width blue bar (touching page edges)
% ---------------------------------------------------------
\newcommand{\FullWidthColorBar}{%
	\noindent
	\hspace*{-\oddsidemargin}%
	\hspace*{-1in}%
	\colorbox{headerblue}{%
		\parbox[c][1.3em][c]{\paperwidth}{%
			\centering\bfseries\color{white}\small \ContestTitle
		}%
	}%
}

% ---------------------------------------------------------
% Header layout
% ---------------------------------------------------------
\newcommand{\ContestHeader}{%
	\begin{minipage}{\textwidth}
		\centering %ICPC|text|uni|Sci|Cul
		\begin{tabular}{@{} m{0.10\textwidth} m{0.45\textwidth} m{0.15\textwidth} m{0.15\textwidth} m{0.15\textwidth} @{}}
			\centering\includegraphics[height=1.6cm]{logo/icpc-logo} &
			\centering\small\bfseries \ContestSeriesName &
			\centering\includegraphics[height=1.6cm]{logo/university-logo} &
			\centering\includegraphics[height=1.6cm]{logo/scientific-logo} &
			\centering\includegraphics[height=1.6cm]{logo/cultural-logo}
		\end{tabular}
		
	\end{minipage}
	
	\vspace{0.3em}
	
	\FullWidthColorBar
}

% ---------------------------------------------------------
% Problem footer variables
% ---------------------------------------------------------
\newcounter{probstartpage}
\newcommand{\currentproblemlabel}{}
\newcommand{\currentproblemname}{}
\newcommand{\currentproblemtotalpages}{}

% ---------------------------------------------------------
% Fancy page styles
% ---------------------------------------------------------
\pagestyle{fancy}

% Main style for problem pages
\fancypagestyle{problemstyle}{%
	\fancyhf{}
	\fancyhead[C]{\ContestHeader}
	\fancyfoot[L]{%
		\rule{\textwidth}{0.4pt}\\[-0.3em]
		\small
		Problem \currentproblemlabel\ -- Page
		\number\numexpr\value{page}-\value{probstartpage}+1\relax\ of \currentproblemtotalpages%
	}%
}

% First page (cover): header only, empty footer
\fancypagestyle{firstpage}{%
	\fancyhf{}
	\fancyhead[C]{\ContestHeader}
}

\pagestyle{problemstyle}

% ---------------------------------------------------------
% Problem environment
%   Usage: \begin{problem}{A}{Micromasters}{1} ... \end{problem}
%   3rd arg = total number of pages of this problem
% ---------------------------------------------------------
\newenvironment{problem}[3]{%
	\clearpage
	\gdef\currentproblemlabel{#1}%
	\gdef\currentproblemname{#2}%
	\gdef\currentproblemtotalpages{#3}%
	\setcounter{probstartpage}{\value{page}}%
	\section*{Problem \currentproblemlabel\ : \currentproblemname}
}{%
	% nothing
}

% ---------------------------------------------------------
% Example table command
% ---------------------------------------------------------
\newcommand{\sampleio}[2]{%
	\begin{center}
		\renewcommand{\arraystretch}{1.2}%
		\begin{tabular}{|p{0.47\textwidth}|p{0.47\textwidth}|}
			\hline
			\rowcolor{headergray}
			\centering\textbf{Standard Input} &
			\centering\textbf{Standard Output} \tabularnewline
			\hline
			{\ttfamily\raggedright #1} &
			{\ttfamily\raggedright #2} \tabularnewline
			\hline
		\end{tabular}
	\end{center}
}

% =========================================================
% Document starts
% =========================================================
\begin{document}
	
	% ---------------------------------------------------------
	% First page (cover)
	% ---------------------------------------------------------
	\thispagestyle{firstpage}
	
	\vspace*{3.5cm}
	\begin{center}
		{\LARGE\bfseries \ContestTitle \par}
		
		\vspace{1.2cm}
		\includegraphics[height=3cm]{logo/icpc-logo}
		
		\vspace{1.5cm}
		{\large Date: May 2025 \par}
		
		\vspace{1.5cm}
		{\bfseries Scientific Committee}\\[0.4em]
		Alice Example \\
		Bob Example \\
		Charlie Example
		
		\vspace{1cm}
		{\bfseries Technical Committee}\\[0.4em]
		Dave Example \\
		Eve Example
		
		\vspace{1cm}
		{\bfseries Executive Committee}\\[0.4em]
		Foo Example \\
		Bar Example
	\end{center}
	
	% ---------------------------------------------------------
	% From now on, use problemstyle
	% ---------------------------------------------------------
	\pagestyle{problemstyle}
	
	% ---------------------------------------------------------
	% Sample Problem A
	% ---------------------------------------------------------
	\begin{problem}{A}{Micromasters}{1}
		Mina is a talented student who refers students to an online program
		called Micromasters. For each referred student, she gets a 10\% discount
		on one course registration. Given the number of referred students, you
		are asked to compute how many courses she can take for free.
		
		\textbf{Input}
		
		The input consists of a single line containing an integer $n$ ($0 \le n \le 1000$),
		the number of students referred by Mina.
		
		\textbf{Output}
		
		Print a single line containing the number of courses Mina can enroll in
		for free.
		
		\textbf{Example}
		
		\sampleio{5}{0}
		
		\sampleio{18}{1}
		
	\end{problem}
	
	% ---------------------------------------------------------
	% Sample Problem B (2 pages)
	% ---------------------------------------------------------
	\begin{problem}{B}{Sample Problem}{2}
		This is another sample problem statement. Write your full English
		description here. There is no box around the description, and there is
		no separate ``Constraints'' section. Any bounds or conditions should be
		explained in the input description if needed.
		
		This is another sample problem statement. Write your full English
		description here. There is no box around the description, and there is
		no separate ``Constraints'' section. Any bounds or conditions should be
		explained in the input description if needed.
		
		This is another sample problem statement. Write your full English
		description here. There is no box around the description, and there is
		no separate ``Constraints'' section. Any bounds or conditions should be
		explained in the input description if needed.
		
		This is another sample problem statement. Write your full English
		description here. There is no box around the description, and there is
		no separate ``Constraints'' section. Any bounds or conditions should be
		explained in the input description if needed.
		
		This is another sample problem statement. Write your full English
		description here. There is no box around the description, and there is
		no separate ``Constraints'' section. Any bounds or conditions should be
		explained in the input description if needed.
		
		
		This is another sample problem statement. Write your full English
		description here. There is no box around the description, and there is
		no separate ``Constraints'' section. Any bounds or conditions should be
		explained in the input description if needed.
		
		
		
		This is another sample problem statement. Write your full English
		description here. There is no box around the description, and there is
		no separate ``Constraints'' section. Any bounds or conditions should be
		explained in the input description if needed.
		
		
		
		This is another sample problem statement. Write your full English
		description here. There is no box around the description, and there is
		no separate ``Constraints'' section. Any bounds or conditions should be
		explained in the input description if needed.
		
		
		
		This is another sample problem statement. Write your full English
		description here. There is no box around the description, and there is
		no separate ``Constraints'' section. Any bounds or conditions should be
		explained in the input description if needed.
		
		
		
		This is another sample problem statement. Write your full English
		description here. There is no box around the description, and there is
		no separate ``Constraints'' section. Any bounds or conditions should be
		explained in the input description if needed.
		
		
		This is another sample problem statement. Write your full English
		description here. There is no box around the description, and there is
		no separate ``Constraints'' section. Any bounds or conditions should be
		explained in the input description if needed.
		
		This is another sample problem statement. Write your full English
		description here. There is no box around the description, and there is
		no separate ``Constraints'' section. Any bounds or conditions should be
		explained in the input description if needed.
		
		
		\textbf{Input}
		
		Describe the input format here, including all necessary conditions and
		bounds.
		
		\textbf{Output}
		
		Describe the required output here.
		
		\textbf{Example}
		
		\sampleio{%
			4 6\\
			0 U\\
			0 D\\
			6 U\\
			3 U%
		}{%
			15%
		}
		
		More explanation, notes, or additional examples can go here to
		demonstrate how a multi-page problem would look.
	\end{problem}
	
\end{document}
